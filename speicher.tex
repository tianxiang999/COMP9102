\subsection{Speicher}

The Speicher system~\cite{} is another KVS that provides rollback protection.
It mainly has two differences compared to ROTE introduced above. First, 
Speicher is a typical \textit{store-then-inc} based system that first stores 
object with pre-incremented counter values into persistent disk memory and increments
counter value thereafter. The second difference lies in the architecture. The ROTE system
uses several assisting severs organized by a trusted group manager while the Speicher system 
uses only localized protection.


\subsubsection{Overview of Speicher}


The update stage of Speicher works as follows. To update a record or add a new 
record in KVS, Speicher first inserts the record into a runtime data structure
named as \textit{memtable}. Then the counter is incremented to the value in 
the stored record. Speicher is a fully asynchronous system that decouples the 
increment operation and the store operation, which means that Speicher batches
the increment operation after a \textit{expected time}. During the expected 
time, the user can wait for the increment of counter to be stable, and can 
abort the update operation if time expires. This mechanism prevents users 
from potential rollback attacks. In case a system crashes between a store
operation and increment operation, the stored record will become invalid as 
the counter value is beyond the older version of the counter stored in disk.




\subsubsection{System Description}






\subsubsection{Limitations and Future Directions}

We claim that the design of Speicher does improve rollback protection and is robust to 
restore at any time of protocol execution. As an SGX based KVS, Speicher also considers 
the limited size of enclaves (around 128MB), and designed a new data structure outside 
the enclave to preserve the integrity. Unfortunately, Speicher may have failed to breakdown 
the overhead by each SGX components. For example, the asynchronous trusted counter mechanism
batches arrived counter increment operations and we expect to see the breakdown to see how 
this mechanism helps improve the throughput.


