\section{Problem}
\label{problem}
In this section, we introduce the SGX architecture with three main operations directly related to our rollback protection designations. Then we describe the adversary model targets in rollback attacks by detailing their assumptions and capabilities.

\subsection{SGX Architecture}
In a standard SGX as specified in~\cite{}, apart from the confidentiality and integrity nature of SGX, there are fundamentally three operations we concern in this report, \textit{i.e.}, the enclave creation, the sealing, and the attestation.
\begin{itemize}
	
	\item \textbf{Enclave creation.}  An enclave is created by the user client. In enclave creation, the client specifies the code to be processed in SGX. Security mechanisms in the processors create a data structure called SGX Enclave Control Structure (SECS) that is stored in a protected memory area. Enclaves' code created by the client cannot contain sensitive data. The start of the enclave is recorded by the processor, reflecting the content of the enclave code as well as the loading a sequence of instructions. The recording of an enclave start is called measurement and it can be used for later attestation. Once an enclave is no longer needed, the OS can terminate it and thus erase its memory structure from the protected memory.
	\item \textbf{Sealing} Enclaves can save confidential data across executions. Sealing is the process to encrypt and authenticate enclave data for persistent storage~\cite{}. All local persistent storage (\textit{e.g.} disk) is controlled by the untrusted OS. For each enclave, the SGX architecture provides a sealing key that is private to the executing platform and the enclave. The sealing key is derived from a Fuse Key (unique to the platform, not known to Intel) and an Identity Key that can be either the Enclave Identity or Signing Identity. The Enclave Identity is a cryptographic hash of the enclave measurement and uniquely identifies the enclave. If data is sealed with Enclave Identity, it is only available to this particular enclave version. The Signing Identity is provided by an authority that signs the enclave prior to its distribution. Data sealed with Signing Identity can be shared among all enclave versions that have been signed with the same Signing Identity.

	\item \textbf{Attestation} Attestation is the process of verifying that certain enclave code has been properly initialized. In local attestation a prover enclave can request a statement that contains measurements of its initialization sequence, enclave code and the issuer key. Another enclave on the same platform can verify this statement using a shared key created by the processor. In remote attestation the verifier may reside on another platform. A system service called Quoting Enclave signs the local attestation statement for remote verification. The verifier checks the attestation signature with the help of an online attestation service that is run by Intel. Each verifier must obtain a key from Intel to authenticate to the attestation service. The signing key used by the Quoting Enclave is based on a group signature scheme called EPID (Enhanced Privacy ID) which supports two modes of attestation: fully anonymous and linkable attestation using pseudonyms~\cite{}. The pseudonyms remain invariant across reboot cycles (for the same verifier). Once an enclave has been attested, the verifier can establish a secure channel to it using an authenticated key exchange mechanism.

\end{itemize}

In this report, protocols described in Section~\ref{solutions} primally utilize these three operations in SGX to provide rollback attack protections.


\subsection{Adversary Model}
In this report, we consider rollback attacks~\cite{}, where the adversary may carry out their attacks in two ways. One is replay, \textit{i.e.}, arbitrarily shut down the system, and replay from a stale state. The other direction is forking attacks~\cite{}, where the adversary attempt to fork the storage system, \textit{e.g.}, by running multiple replicas of the storage system.

The goal of methods specified in Section~\ref{solutions} is to guarantee the data integrity, confidentiality, and freshness towards rollback attacks based on SGX architecture. Note that for different methods, different level of adversary's strength is considered, which is listed and compared in Section~\ref{solutions}.




