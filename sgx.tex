\subsection{SGX Counter}


\subsubsection{Limitations}
Though being as a selective feature in SGX architecture, it has strict memory constraints and performs slow during experimental tests~\cite{}. 

The SGX Monotonic Counter updates take 80-250 ms and reads 60-140 ms. When an enclave needs to persistently store an updated state, it can increment a counter, include the counter value and identifier to the sealed data, and verify integrity of the stored data based on counter value at the time of unsealing. However, such approach may wear out the used non-volatile memory. Assuming a system that updates one of the enclaves on the same platform once every 250 ms, the non-volatile memory used to implement the counter wears out after approximately one million writes, making the counter functionality unusable after a couple of days of continuous use. Even with a modest update rate of one increment per minute, the counters are exhausted in two years. Thus, SGX counters are unsuitable for systems where state updates are frequent and continuous. Additionally, since the non-volatile memory used to store the counters resides outside the processor package, the mechanism is likely vulnerable to bus tapping and flash mirroring attacks~\cite{}.

Note that SGX also provides the SGX trusted time feature for checking the timestamp of one stored data record. However, including a timestamp to each sealed data version only allows an enclave to distinguish which out of two seals is more recent, enclaves cannot identify if the sealed data provided by the OS is fresh and latest.





