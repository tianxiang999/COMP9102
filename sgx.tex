

\subsection{Monotonic Counter}

In the latest version of SGX, Intel releases the abstraction of \textit{monotonic counter} which can be utilized to 
protect against rollback attacks that replay objects~\cite{alder2018migrating}. When calling the 
\textit{sgx\_cerate\_monotonic\_counter} function from the SGX library, it automatically creates a 
limited number of monotonic counters (MC) for each enclave instance on the platform. The MC is shared 
among all the instances who run the same code. Upon creating a new MC, it gets written to the non-volatile
untrusted memory through a secure channel, preventing malicious OS or hypervisor from changing the counter 
value or replaying value~\cite{shinde2017panoply}.

With the help of monotonic counter, a basic approach is to store
the state of objects with the counter into persistent memory and check the counter value 
each time request for the object. This approach is trivially feasible to to address 
rollback attacks but suffer from a significant weakness. The performance of SGX monotonic 
counter is not well documented~\cite{strackx2016ariadne}. Many prior work did experiments on the \textit{write}
performance of monotonic counter and found that writes of counter values to persistent
memory is slow (around 10 writes a second). This weakness largely limits the performance in 
current high throughput KVS systems such as Redis~\cite{paksula2010persisting} and Apache Zookeeper~\cite{hunt2010zookeeper}.
Thus, directly applying monotonic counter to preserve freshness is impractical.



\subsubsection{Limitations and Future Directions}
Though being as a selective feature in SGX architecture, it has strict memory constraints and performs slow during experimental tests~\cite{alder2018migrating}. 

The SGX Monotonic Counter updates take 80-250 ms and reads 60-140 ms. When an enclave needs to persistently store an updated state, it can increment a counter, include the counter value and identifier to the sealed data, and verify integrity of the stored data based on counter value at the time of unsealing. However, such approach may wear out the used non-volatile memory. Assuming a system that updates one of the enclaves on the same platform once every 250 ms, the non-volatile memory used to implement the counter wears out after approximately one million writes, making the counter functionality unusable after a couple of days of continuous use. Even with a modest update rate of one increment per minute, the counters are exhausted in two years. Thus, SGX counters are unsuitable for systems where state updates are frequent and continuous. Additionally, since the non-volatile memory used to store the counters resides outside the processor package, the mechanism is likely vulnerable to bus tapping and flash mirroring attacks~\cite{skorobogatov2016bumpy}.

Note that SGX also provides the SGX trusted time feature for checking the timestamp of one stored data record. However, including a timestamp to each sealed data version only allows an enclave to distinguish which out of two seals is more recent, enclaves cannot identify if the sealed data provided by the OS is fresh and latest.

However, the idea of counter increment technique does exist and recent papers~\cite{bailleu2019speicher, enclavedb:sp18, matetic2017rote} have 
shown that users can indeed benefit from such protection against rollback attacks.
Basically, there are two kinds of solutions for counter-based rollback protection.
The first technique is \textit{inc-then-store}, where the enclave first increments the counter
value and then stores the sealed object together with the incremented value 
to persistent memory. This approach guarantees that the platform can detect any rollback 
of stored objects by checking the latest counter value. Even when the system crashes after 
the rollback, the enclave can restart and check the counter value in the persistent memory,
and restore to the updated state (value) of the counter without breaking the protection mechanism.
But if the system fails at runtime, the \textit{inc-then-store} can not recover because 
the counter has a future value while the latest stored object in persistent memory has a smaller 
counter value. Due to the deterministic increase of the counter, the system cannot recover from 
system crash.

The second approach is \textit{store-then-inc}, where the enclave first stores the object with 
an incremented counter value to persistent memory and increments the counter thereafter.
This technique can greatly improve the throughput of KVS because the enclave no longer needs to 
wait for a complete process of incrementing counter and writing the value to persistent memory,
instead, the enclave can batch the increment operation of counters and avoid the bottleneck 
of writing counters to persistent memory (80 $\sim$ 250 ms). Another benefit of this technique is 
that if the system crashes, the system can recover from the failure by referring to the counter 
value in persistent memory, even in the runtime of protocol. Because the system can detect a future 
value of counter from persistent memory, by referring to the current state of the counter, the 
system can check for the missing records and ignore the records with future counter value.

The two techniques are both practical but have different drawbacks which should be taken into 
consideration in system build up. The drawbacks of \textit{inc-then-store} technique mainly include: 
(a) it cannot recover from runtime failure and (b) it has relatively slow throughput as each seal operation 
should wait for the write of counter. For the \textit{store-then-inc} technique, it has higher throughput 
but may suffer from replay attacks~\cite{brenner2016securekeeper}. 
