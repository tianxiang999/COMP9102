\section{Introduction}
With the growth in cloud computing adoption, online data stored in data centers is growing at an ever increasing rate~\cite{}. Modern online services ubiquitously use persistent key-value (KV) storage systems to store data with a high degree of reliability and performance~\cite{}. Therefore, persistent KV stores have become a fundamental part of the cloud infrastructure.

At the same time, the risks of security violations in storage systems have increased significantly for the third-party cloud computing infrastructure~\cite{}. Modern data processing services hosted in cloud environments are under constant attack from malicious entities such as database administrators, server administrators, hackers who exploit bugs in the operating system or hypervisor, and even nation states. This results in frequent data breaches that reduce trust in online services. Semantically secure encryption can provide strong and efficient protection for data at rest and in transit, but this is not sufficient because data processing systems decrypt sensitive data in memory during query processing. In an untrusted environment, an attacker can compromise the security properties of the stored data and query operations. In fact, many studies show that software bugs, configuration errors, and security vulnerabilities pose a serious threat to storage systems~\cite{}.

However, securing a storage system is quite challenging because modern storage systems are quite complex~\cite{}. Thereby, the enforcement of security policies needs to be carried out by various layers in the system stack, which could expose the data to security vulnerabilities. Furthermore, since the data is stored outside the control of the data owner, the third-party storage platform provides an additional attack vector. The clients currently have limited support to verify whether the third-party operator, even with good intentions, can handle the data with the stated security guarantees.

A approach to enable secure query processing is to trusted execution environments (TEEs), such as Intel SGX~\cite{} or ARM TrustZone~\cite{}, provide an appealing approach to build secure systems. Enclaves can protect sensitive data and code, even from powerful attackers that control or have compromised the operating system and the hypervisor on a host machine. While enclaves can mitigate several attacks, using them requires careful refactoring of applications into trusted and untrusted components to achieve desired security and privacy goals. Furthermore, ensuring high level security properties such as confidentiality, integrity, and freshness requires additional logic to protect secrets when they leave the enclave and verify their integrity when they are read. This task is relatively simple in applications such as password checkers, key management systems and simpler data processing frameworks. In fact, given the importance of security threats in the cloud, there is a recent surge in leveraging TEEs for shielded execution of applications in the untrusted infrastructure~\cite{}. Shielded execution aims to provide strong security properties using a hardware-protected secure memory region or enclave.

While SGX can be considered as a big step forward towards trustworthy cloud computing, some attack vectors nevertheless remain. One important open issue are rollback and forking attacks on stateful applications that make use of persistent storage. Whereas SGX provides mechanisms against main- memory replay attacks, persistent storage is not under the direct control of SGX and therefore harder to secure. The need to handle system restarts, operating system crashes, and power outages makes a completely secure solution for state continuity difficult to achieve. 

In this report, we mainly survey three state-of-art methods in solving the secure key-value storage with rollback attacks protections, \textit{i.e.}, Monotonic Counter for SGX, ROTE system, and Speicher system. We separately describe their inner designations and their limitations together with our insights for future directions.
