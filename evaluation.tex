\section{Evaluation Plan}
To evaluate the effectiveness of the above-mentioned protection mechanisms and see 
how our proposed ideas may help improve the threat model, we plan to analyze the following 
features in the future.

\noindent\textbf{Performance of Update and reboot. }
Either a protection mechanism is used \textit{inc-then-store} or \textit{store-then-inc}, 
it's significant to see the practicality in a large-scale KVS. In other word, by integrating 
the rollback protection, the expected performance of update stage, including both the I/O throughput and
end-to-end communication latency should be well studied. If there is moderate overhead compared 
to the original design (e.g., Redis or Zookeeper), then the plan is considered to be both 
cost-efficient and protection-effective. However, if the update stage cost significant
overhead, then it is impractical to be deployed in real world.

The recovery of KVS (i.e., restore) from a system crash, reboot or failure should be 
measured as well. As the integrity of SGX-based mechanism is usually assumed to be immune to 
parties outside the enclave, the only way to break the security property of these systems 
is to completely control all the parties engaged in version check. Thus the defense of such 
vulnerability might take extra overhead. We should evaluate how the recovery module effects 
the effectiveness of KVS systems, such as manually reboot the system or cut off power.

\noindent\textbf{SGX overhead. }
The monotonic counter is slow, as introduced in both Background and \S\ref{problem}.
We need to evaluate the overhead of using SGX isolation. More precisely, we should 
breakdown the I/O overhead of entering/exiting SGX enclaves, the latency of counter mechanism (e.g., Speicher's asynchronous 
counter increment), and the \textit{Read}/\textit{Write} throughput with limited enclave size.