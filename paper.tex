% This is sigproc-sp.tex -FILE FOR V2.6SP OF ACM_PROC_ARTICLE-SP.CLS
% OCTOBER 2002
%
% It is an example file showing how to use the 'acm_proc_article-sp.cls' V2.6SP
% LaTeX2e document class file for Conference Proceedings submissions.
% ----------------------------------------------------------------------------------------------------------------
% This .tex file (and associated .cls V2.6SP) *DOES NOT* produce:
%       1) The Permission Statement
%       2) The Conference (location) Info information
%       3) The Copyright Line with ACM data
%       4) Page numbering
%
%  However, both the CopyrightYear (default to 2002) and the ACM Copyright Data
% (default to X-XXXXX-XX-X/XX/XX) can still be over-ridden by whatever the author
% inserts into the source .tex file.
% e.g.
% \CopyrightYear{2003} will cause 2003 to appear in the copyright line.
% \crdata{0-12345-67-8/90/12} will cause 0-12345-67-8/90/12 to appear in the copyright line.
%
% ---------------------------------------------------------------------------------------------------------------
% It is an example which *does* use the .bib file (from which the .bbl file
% is produced).
% REMEMBER HOWEVER: After having produced the .bbl file,
% and prior to final submission,
% you need to 'insert'  your .bbl file into your source .tex file so as to provide
% ONE 'self-contained' source file.
%
% Questions regarding SIGS should be sent to
% Adrienne Griscti ---> griscti@acm.org
%
% Questions/suggestions regarding the guidelines, .tex and .cls files, etc. to
% Gerald Murray ---> murray@acm.org
%
% For tracking purposes - this is V2.6SP - OCTOBER 2002

\documentclass{acm_proc_article-csis8101}

\begin{document}
%
% --- Author Metadata here ---
\conferenceinfo{COMP9102 A Survey on Rollback Attack Protections for Key-Value Storage}{2019, CS Dept, HKU}
%\setpagenumber{50}
%\CopyrightYear{2002} % Allows default copyright year (2002) to be over-ridden - IF NEED BE.
%\crdata{0-12345-67-8/90/01}  % Allows default copyright data (X-XXXXX-XX-X/XX/XX) to be over-ridden.
% --- End of Author Metadata ---

\title{A Survey on Rollback Attack Protections\\for Key-Value Storage}
%
% You need the command \numberofauthors to handle the "boxing"
% and alignment of the authors under the title, and to add
% a section for authors number 4 through n.

\author{
Ke Zhang\\
3030058805\\
Department of Computer Science\\
University of Hong Kong \\
Pokfulam Road, Hong Kong\\
\texttt{kzhang2@cs.hku.hk}
\and Tianxiang Shen\\
3030058776\\
Department of Computer Science\\
University of Hong Kong \\
Pokfulam Road, Hong Kong \\
\texttt{txshen2@cs.hku.hk}
}

\numberofauthors{2}
%
\maketitle
\begin{abstract}
In this report, based on the mature secure enclave architecture Intel Software Guard Extensions (SGX), we survey three state-of-art secure key-value storage methods against rollback attacks. For each approach, we discuss respective practicality, efficiency, and secure guarantee level regarding to their inner mechanisms and experimental performances. Furthermore, we point out possible improving directions for methods we discuss in this report.

\end{abstract}



\section{Introduction}
With the growth in cloud computing adoption, online data stored in data centers is growing at an ever 
increasing rate~\cite{bhatotia2012shredder}. Modern online services ubiquitously use persistent key-value (KV) store systems to store data with a high degree of reliability and performance~\cite{bailleu2019speicher}. Therefore, persistent KV stores have become a fundamental part of the cloud infrastructure.

At the same time, the risks of security violations in storage systems have increased significantly for the third-party cloud computing infrastructure~\cite{santos2009towards}. Modern data processing services hosted in cloud environments are under constant attack from malicious entities such as database administrators, server administrators, hackers who exploit bugs in the operating system or hypervisor, and even nation states. This results in frequent data breaches that reduce trust in online services. Semantically secure encryption can provide strong and efficient protection for data at rest and in transit, but this is not sufficient because data processing systems decrypt sensitive data in memory during query processing. In an untrusted environment, an attacker can compromise the security properties of the stored data and query operations. In fact, many studies show that software bugs, configuration errors, and security vulnerabilities pose a serious threat to storage systems~\cite{gunawi2014bugs}.

It is quite challenging to secure a storage system because modern storage systems are quite complex~\cite{lu2013study}. Thereby, the enforcement of security policies needs to be carried out by various layers in the system stack, which could expose the data to security vulnerabilities. Furthermore, since the data is stored outside the control of the data owner, the third-party storage platform provides an additional attack vector. The clients currently have limited support to verify whether the third-party operator, even with good intentions, can handle the data with the stated security guarantees.

An approach to enable secure query processing is trusted execution environments (TEEs), such as Intel Software Guard Extensions 
(SGX)~\cite{sgxexplained} or ARM TrustZone~\cite{winter2008trusted}, which provide an appealing approach to build secure systems. Enclaves can protect sensitive data and code, even from powerful attackers that control or have compromised the operating system and the hypervisor on a host machine. While enclaves can mitigate several attacks, using them requires careful refactoring of applications into trusted and untrusted components to achieve desired security and privacy goals. Furthermore, ensuring high level security properties such as confidentiality, integrity, and freshness requires additional logic to protect secrets when they leave the enclave and verify their integrity when they are read. This task is relatively simple in applications such as password checkers, key management systems and simpler data processing frameworks. In fact, given the importance of security threats in the cloud, there is a recent surge in leveraging TEEs for shielded execution of applications in the untrusted infrastructure. Shielded execution aims to provide strong security properties using a hardware-protected secure memory region or enclave.

While SGX can be considered as a big step forward towards trustworthy cloud computing, some attack vectors nevertheless remain. One important open issue are rollback and forking attacks on stateful applications that make use of persistent storage. Whereas SGX provides mechanisms against main- memory replay attacks, persistent storage is not under the direct control of SGX and therefore harder to secure. The need to handle system restarts, operating system crashes, and power outages makes a completely secure solution for state continuity difficult to achieve. 

In this report, we mainly survey three state-of-art methods in solving the secure key-value storage with rollback attacks protections, \textit{i.e.}, Monotonic Counter for SGX, ROTE system, and Speicher system. We separately describe their inner designations and their limitations together with our insights for future directions.

\section{Background}
\subsection{SGX Overview}
\subsubsection{Secure Enclave}
A secure enclave is a set of software and hardware features that together provide an isolated execution environment to enable a set of strong security guarantees for applications running inside the enclave. Enclave allows user-level as well as Operating System~(OS) code to define private regions of memory, whose contents are protected and unable to be either read or saved by any process outside the enclave itself, including processes running at higher privilege levels~\cite{}. 

%Primally, secure enclaves can provide confidentiality, integrity, and attestation. Confidentiality guarantees that an adversary outside of the enclave cannot inspect the state of execution inside the enclave, even if they compromise the operating system or correctness of the computation running inside the enclave even if the operating system has been compromised or a user attempts to subvert the execution of the program inside the enclave. Finally, hardware-based attestation provides an unforgeable proof that enables a remote party to verify what has run inside the enclave even if they don’t have physical access to the machine. A secure enclave thus provides a powerful cornerstone for secure computing and development of secure systems in general. 


Intuitively, secure enclave fundamentally ensures the correctness and isolation in executing given process. The confirmation of input data freshness is hard to achieve, especially when the enclave encounters crash or restart. There are several wildly used secure enclave services~\cite{}, one of the most popular security architectures is Intel Software Guard Extensions (SGX)~\cite{}. However, a mature secure enclave designation as SGX still shows unsatisfied performance towards rollback attacks. In this report, we focus on the SGX architecture and its existing promotions in proposing protection against rollback attacks.

\subsubsection{SGX Architecture}
In a standard SGX as specified in~\cite{}, apart from the confidentiality and integrity nature of SGX, there are fundamentally three operations we concern in this report, \textit{i.e.}, the enclave creation, the sealing, and the attestation.
\begin{itemize}
	
	\item \textbf{Enclave creation.}  An enclave is created by the user client. In enclave creation, the client specifies the code to be processed in SGX. Security mechanisms in the processors create a data structure called SGX Enclave Control Structure (SECS) that is stored in a protected memory area. Enclaves' code created by the client cannot contain sensitive data. The start of the enclave is recorded by the processor, reflecting the content of the enclave code as well as the loading a sequence of instructions. The recording of an enclave start is called measurement and it can be used for later attestation. Once an enclave is no longer needed, the OS can terminate it and thus erase its memory structure from the protected memory.
	\item \textbf{Sealing} Enclaves can save confidential data across executions. Sealing is the process to encrypt and authenticate enclave data for persistent storage~\cite{}. All local persistent storage (\textit{e.g.} disk) is controlled by the untrusted OS. For each enclave, the SGX architecture provides a sealing key that is private to the executing platform and the enclave. The sealing key is derived from a Fuse Key (unique to the platform, not known to Intel) and an Identity Key that can be either the Enclave Identity or Signing Identity. The Enclave Identity is a cryptographic hash of the enclave measurement and uniquely identifies the enclave. If data is sealed with Enclave Identity, it is only available to this particular enclave version. The Signing Identity is provided by an authority that signs the enclave prior to its distribution. Data sealed with Signing Identity can be shared among all enclave versions that have been signed with the same Signing Identity.

	\item \textbf{Attestation} Attestation is the process of verifying that certain enclave code has been properly initialized. In local attestation a prover enclave can request a statement that contains measurements of its initialization sequence, enclave code and the issuer key. Another enclave on the same platform can verify this statement using a shared key created by the processor. In remote attestation the verifier may reside on another platform. A system service called Quoting Enclave signs the local attestation statement for remote verification. The verifier checks the attestation signature with the help of an online attestation service that is run by Intel. Each verifier must obtain a key from Intel to authenticate to the attestation service. The signing key used by the Quoting Enclave is based on a group signature scheme called EPID (Enhanced Privacy ID) which supports two modes of attestation: fully anonymous and linkable attestation using pseudonyms~\cite{}. The pseudonyms remain invariant across reboot cycles (for the same verifier). Once an enclave has been attested, the verifier can establish a secure channel to it using an authenticated key exchange mechanism.

\end{itemize}

In this report, protocols described in Section~\ref{solutions} primally utilize these three operations in SGX to provide rollback attack protections.

\subsubsection{SGX Counter}

Intel has recently added support for monotonic counters~(MC)~\cite{} as an optional SGX feature. The Monotonic Counter can be utilized by enclave developers for rollback attack protection. 

SGX supports creating a limited number of MCs for each enclave. Monotonic counters are shared among enclaves that have the same code. An enclave can query availability of counters from the Platform Service Enclave (PSE). If supported, the enclave can create up to 256 counters. The default owner policy encompasses that only enclaves with the same signing key may access the counter. Counter creation operation returns an identifier that is a combination of the Counter ID and a nonce to distinguish counters created by different entities. On creating a MC, it gets written to the non-volatile memory in the platform. The enclave must store the counter identifier to access it later, as there is no API call to list existing counters. After a successful counter creation, an enclave can increment, read, and delete the counter. Because each enclave shares the same value of the monotonic counters, it guarantees the verification for data freshness. In other words, only when an enclave preserves the same counter value as the others in the platform, its reserving data are the latest. Also, when one enclave encounters crash or reboot, it can recover data with the help of monotonic counters shared in the platform. 

According to the SGX API documentation~\cite{}, counter operations involve writing to a non-volatile memory. Repeated write operations can cause the memory to wear out, and thus the counter increment operations may be rate limited. 


\subsection{Rollback Attack}

Rollback attacks remain a potential secure problem in secure enclave. In a rollback attack, attackers replace the latest data with an older version without being identified by the system. 

Data integrity violation through rollback attacks can have severe implications. Consider, for example, a financial application implemented as an enclave. The enclave repeatedly processes incoming transactions at high speed and maintains an account balance for each user or a history of all transactions in the system. If the adversary manages to revert the enclave to its previous state, the maintained account balance or the queried transaction history does not match the executed transactions.

In reality, enclaves cannot easily detect this replay, because the processor is unable to maintain persistent state across enclave executions that may include reboots or crash. Another way to carry out rollback attacks in secure enclaves is to create multiple instances of a same process and route update requests to one instance and read requests to the other. Due to the characteristic of secure enclave, the instances are indistinguishable to remote clients or OS.

To avoid rollback attacks, most commonly considered direction is to record the time related information for every state change. In this paper, we mainly discuss three designations in rollback attacks protection built on the SGX architecture. The goal of methods specified in Section~\ref{solutions} are to guarantee the data integrity, confidentiality, and freshness towards rollback attacks based on SGX architecture. Note that for different methods, different level of adversary's strength is considered, which is listed and compared in Section~\ref{solutions}.


\section{Problem}\label{problem}

As the infrastructure of cloud computing grows rapidly, storage service providers use Key-Value Stores (KVS) 
in data centers to persist user data, with high throughput and low end-to-end communication latency~\cite{}.
Many users store their sensitive data (e.g., password, medical record) in these systems, while the protection 
of these data is not enough. 
Specifically, there are three dominant security properties in KVS: confidentiality, integrity and freshness.
(a) \textbf{Confidentiality} is to ensure that other unauthorized parties (e.g., malicious OS) cannot read the plaintext 
data of personal record in KVS. (b) \textbf{Integrity} is the property that the typical \textit{read} and \textit{write} 
operations of KVS cannot be tampered with, such as the changes to records in persistent storage. (c) \textbf{Freshness}
is the ability to detect stale state of data, in case a malicious KVS returns an older version of a request record.

Intel Software Guard eXtension (SGX), a popular security hardware on commodity available Intel CPUs, is promising 
to provide the first two security properties in KVS~\cite{}. SGX provides an abstraction of secure enclaves, which is a secured 
memory zone isolated from untrusted memories. By sealing enclave objects with secret SGX keys to untrusted memory (i.e., 
persistent storage on host) and unsealing encrypted objects to enclave, SGX ensures that in-enclave data is unavailable 
from the outside, even with a malicious OS or hypervisor~\cite{}.

Unfortunately, the freshness cannot be guaranteed by simply running KVS on SGX-enabled hosts. 
The problem lies in the lack of version check when an enclave loads objects from untrusted memory.
Figure \ref{fig:rollback} shows a typical rollback attack in a local scenario. The enclave calls the \textit{write} 
operation of KVS twice to store two different key-value pairs, respectively. When the enclave
requests for the latest value by calling \textit{read}, the attacker returns a previous version of value 
to the enclave. Since the enclave can only verify source of the returned object from the correct platform
through local attestation, the incorrect returned object cannot be detected by KVS users.

\begin{figure}
    \centering
        \includegraphics[width=.50\textwidth]{rollback.png}
        \caption[title]{An example of rollback attack towards KVS on SGX-enabled host. 
        The malicious OS returns an older version of value in KVS and the trusted enclave (in gray)
        cannot detect it. The records should be sealed/unsealed but we omit these operations for simplicity.}
        % \caption[title]{DAENet's scalability with increasing number of nodes.}
        \label{fig:rollback}
\end{figure}

To formalize, in addition to leverage the protection of SGX, we should also develop a freshness protection 
mechanism to protect against rollback attacks that replay old state of objects. In other word, we aim to expand 
the security protection of SGX from trusted volatile memory of enclaves to untrusted non-volatile memory of 
the outside, even when the system reboot, crash or during migration.

\section{Solutions}
In this section, we mainly introduce three state-of-art solutions in solving the problem we mention in \S\ref{problem}. 
We separately describe their motivation, inner designations, and respective improving directions in detail.



\subsection{Monotonic Counter}

In the latest version of SGX, Intel releases the abstraction of \textit{monotonic counter} which can be utilized to 
protect against rollback attacks that replay objects~\cite{}. When calling the 
\textit{sgx\_cerate\_monotonic\_counter} function from the SGX library, it automatically creates a 
limited number of monotonic counters (MC) for each enclave instance on the platform. The MC is shared 
among all the instances who run the same code. Upon creating a new MC, it gets written to the non-volatile
untrusted memory through a secure channel, preventing malicious OS or hypervisor from changing the counter 
value or replaying value~\cite{}.

With the help of monotonic counter, a basic approach is to store
the state of objects with the counter into persistent memory and check the counter value 
each time request for the object. This approach is trivially feasible to to address 
rollback attacks but suffer from a significant weakness. The performance of SGX monotonic 
counter is not well documented~\cite{}. Many prior work did experiments on the \textit{write}
performance of monotonic counter and found that writes of counter values to persistent
memory is slow (around 10 writes a second). This weakness largely limits the performance in 
current high throughput KVS systems such as Redis~\cite{} and Apache Zookeeper~\cite{}.
Thus, directly applying monotonic counter to preserve freshness is impractical.



\subsubsection{Limitations}
Though being as a selective feature in SGX architecture, it has strict memory constraints and performs slow during experimental tests~\cite{}. 

The SGX Monotonic Counter updates take 80-250 ms and reads 60-140 ms. When an enclave needs to persistently store an updated state, it can increment a counter, include the counter value and identifier to the sealed data, and verify integrity of the stored data based on counter value at the time of unsealing. However, such approach may wear out the used non-volatile memory. Assuming a system that updates one of the enclaves on the same platform once every 250 ms, the non-volatile memory used to implement the counter wears out after approximately one million writes, making the counter functionality unusable after a couple of days of continuous use. Even with a modest update rate of one increment per minute, the counters are exhausted in two years. Thus, SGX counters are unsuitable for systems where state updates are frequent and continuous. Additionally, since the non-volatile memory used to store the counters resides outside the processor package, the mechanism is likely vulnerable to bus tapping and flash mirroring attacks~\cite{}.

Note that SGX also provides the SGX trusted time feature for checking the timestamp of one stored data record. However, including a timestamp to each sealed data version only allows an enclave to distinguish which out of two seals is more recent, enclaves cannot identify if the sealed data provided by the OS is fresh and latest.

However, the idea of counter increment technique does exist and recent papers~\cite{} have 
shown that users can indeed benefit from such protection against rollback attacks.
Basically, there are two kinds of solutions for counter-based rollback protection.
The first technique is \textit{inc-then-store}, where the enclave first increments the counter
value and then stores the sealed object together with the incremented value 
to persistent memory. This approach guarantees that the platform can detect any rollback 
of stored objects by checking the latest counter value. Even when the system crashes after 
the rollback, the enclave can restart and check the counter value in the persistent memory,
and restore to the updated state (value) of the counter without breaking the protection mechanism.
But if the system fails at runtime, the \textit{inc-then-store} can not recover because 
the counter has a future value while the latest stored object in persistent memory has a smaller 
counter value. Due to the deterministic increase of the counter, the system cannot recover from 
system crash.

The second approach is \textit{store-then-inc}, where the enclave first stores the object with 
an incremented counter value to persistent memory and increments the counter thereafter.
This technique can greatly improve the throughput of KVS because the enclave no longer needs to 
wait for a complete process of incrementing counter and writing the value to persistent memory,
instead, the enclave can batch the increment operation of counters and avoid the bottleneck 
of writing counters to persistent memory (80 $\sim$ 250 ms). Another benefit of this technique is 
that if the system crashes, the system can recover from the failure by referring to the counter 
value in persistent memory, even in the runtime of protocol. Because the system can detect a future 
value of counter from persistent memory, by referring to the current state of the counter, the 
system can check for the missing records and ignore the records with future counter value.

The two techniques are both practical but have different drawbacks which should be taken into 
consideration in system build up. The drawbacks of \textit{inc-then-store} technique mainly include: 
(a) it cannot recover from runtime failure and (b) it has relatively slow throughput as each seal operation 
should wait for the write of counter. For the \textit{store-then-inc} technique, it has higher throughput 
but may suffer from replay attacks~\cite{}. 

\subsection{ROTE}

To overcome the slowness of SGX Monotonic Counters and provide stable persistent 
rollback attack protections, ROTE~\cite{} is proposed as a distributed trusted 
counter service based on a consensus protocol.

\subsubsection{Overview of ROTE}
\label{overview_rote}
ROTE is a \textit{inc-then-store} based system that protects against rollback attacks.
To overcome the low throughput of monotonic counter increment, ROTE uses a distributed 
secure counter storage to help verify the version of a target enclave. The intuition 
behind is simple, that a single SGX-enabled platform is difficult to prevent rollback
attacks but many platforms can work together to assist the process of verification.
The assisting servers are incentive to do this job as they can also benefit from 
such protection.

With the assistance of a group of servers, ROTE assumes a strong adversary  
that can either control the OS of the target platform or any of the assisting platforms.
The adversary can break the protection of SGX and even act as a network-level administrator
that controls the interactive communication in the network by delaying, replaying or 
revising network packets. However, as a \textit{inc-then-store} based system, ROTE 
assumes no tolerance of some of the platform crashes and by default no crash will happen
in a protocol run. If crash tolerance is required, then a \textit{store-then-inc}
technique is required, and even the system should support both of the techniques to 
allow users choose by their tolerance of crash.

The update stage of ROTE works as follows. A client first triggers a counter increment 
in local enclave, the enclave increments the counter (initialized as zero) in runtime
memory, signs the counter value and sends the signed counter value to all assisting 
servers. Upon receiving the signed counter value, each assisting server updates the 
value of targeting (client's) counter table in memory and sends back their state 
of counter value. Note that the value is temporarily stored in memory 
and not sealed to disk to avoid endless propagation. When the client receives $q$
feedbacks, it compares the value and returns ACks if the value matches its own.
Then, the client can ensure that the version is correct, seal the current counter 
value and the object to disk. 

ROTE also develops a protocol to recover from system reboot/failure, and a distributed 
mechanism to securely store and compare counter values in remote assisting servers.
With a strong network adversary model, ROTE protects
against both network partitioning and replay attacks. The update protocol, 
recover protocol and distributed secure storage mechanism work together to make ROTE 
a robust KVS system that provides protection against rollback attacks.



\subsubsection{System Protocols}
\begin{figure}
    \centering
        \includegraphics[width=.45\textwidth]{rote_sys}
        \caption[title]{The ROTE system architecture.}
        \label{fig:rote_sys}
\end{figure}

Figure \ref{fig:rote_sys} shows ROTE system architecture. Every user application running on platforms matches an Application-Specific Enclave (ASE). The ROTE system provides a Rollback Enclave (RE) and a ROTE library for ASEs as a rollback protection service. The RE maintains a Monotonic Counter (MC), increases it for every ASE update, distributes it to REs running on assisting platforms, and includes the counter value to its own sealed data. 

For easier descriptions, we denote $n$ as the number of assisting platforms, $f$ as the number of compromised processors, and $u$ as the tolerance of  unreachable assisting platforms when the system proceeds write/read operation. These three parameters have a dependency $n = f + 2u + 1$ to fulfill the data integrity, attestation and freshness of the ROTE system consensus protocols. In the ROTE system, there are three protocols designed for ASE state update, RE restart, and ASE start/read. Specifically, messages transmitted in the ROTE system are all encrypted with respective session keys for data confidentiality concerns. We respectively specify them as follows.
\begin{itemize}
	\item \textbf{ASE State Update Protocol} When an ASE is ready to update its state, it starts the state update protocol. This protocol can be regarded as a modification of the Echo broadcast~\cite{}. 
	\begin{enumerate}
		\item The ASE triggers a counter increment using the RE.
		\item The RE increments its own MC, and signs the MC. 
		\item The RE sends the signed counter to all REs in the protection group.
		\item Upon receiving the signed MC, each RE updates its group counter table kept in the runtime memory without sealing the received data.
		\item The REs that received the counter saves the echo in runtime memory and broadcasts an echo message containing the received signed MC.
		\item After receiving $q=u+f+1$ echos, the RE returns the echos to their senders.
		\item Upon receiving back the echo, each RE finds the self-sent echo in its memory. Then every RE checks if the value from echo, from the group counter table, and from the target RE are equal. If these three values match each other, the RE replies with a final ACK message.
		\item After receiving $q=u+f+1$ final ACKs, the RE seals its own state together with the MC value to the disk.
		\item The RE returns the incremented ASE counter value. The ASE can now safely perform the state update. The ASE saves the counter value to its runtime memory and seal its state with the counter.
	\end{enumerate}
	\item \textbf{RE Restart Protocol} The goal of the protocol is to allow the RE to join the existing protection group, retrieve its counter value and the MC values of the other nodes. It supports at most $u$ REs restart simultaneously. 
	\begin{enumerate}
		\item Reset cryptology keys and update system configuration informations
		\item The RE queries the OS for the sealed state.
		\item The RE unseals the state (if received) and extracts the MC.
		\item \label{s}The RE sends a request to all other REs in the same protection group to retrieve its MC.
		\item The assisting REs check their group counter table. If the MC is found, the enclaves reply with the signed MC. Additionally, the target RE receives the complete table all signed MCs from assisting REs.
		\item \label{e}When the RE receives $q=u+f+1$, where $q \ge n/2$ with at least $f+1$ counter values not zero, responses from the group, it selects the maximum value and verifies the signature. For each assisting RE, the target RE picks the highest MC and updates its own group counter table with the value. If the obtained counter value equals to the unsealed data, the unsealed state can be accepted.
		\item The RE stores and seals the updated state to both persistent and runtime memory.
		\end{enumerate}
	\item \textbf{ASE Start/Read Protocol} When an ASE needs to verify the freshness of its state, it performs this protocol.
	\begin{enumerate}
		\item The ASE queries the OS for the sealed data.
		\item The ASE unseals the state (if received) and obtains a counter value from it.
		\item The ASE issues a request to the local RE to retrieve its latest ASE counter value.
		\item To verify the freshness of its runtime state, the RE performs the steps \ref{s}-\ref{e} from the RE Restart protocol, to obtain the latest MC from the network. If the obtained MC does not match the MC residing in the memory, the RE must abort and be restarted. If the values match, the current data is fresh and the RE can continue normal operation.
		\item If all verification checks are successful, the RE returns a value from the local ASE counter table.
		\item The ASE compares the received counter value to the one obtained from the sealed data.
	\end{enumerate}
\end{itemize} 

Notice that in ROTE system defines a required quorum with size $q=u+f+1$ for secure consensus. The reason behind $q$ value is that if the counter is successfully written to $q = f + u + 1$ nodes, there always exists at least $u + 1$ honest platforms in the group that have the latest counter value in the memory. Because counter reading requires the same number of responses, at least one correct counter value is obtained upon reading. If the quorum cannot be satisfied in either the state update protocol or any counter retrieval, ROTE turns to halt and try to perform the same operation again.







%\begin{figure}
%    \centering
%        \includegraphics[width=.45\textwidth]{states}
%        \caption[title]{Transition diagram showing enclave execution states using an ideal secure counter storage functionality.}
%        \label{fig:states}
%\end{figure}


\subsubsection{Limitations}

As is mentioned above, ROTE leverages \textit{inc-then-store} counter increment 
technique as the foundation to defend against rollback attacks. The bottleneck 
of such technique still exists in ROTE: the crash during protocol run can totally 
ruin the system and prevent the system from recovery. 

We propose two future directions for further improving ROTE's trust model by enabling 
crash recovery between sealing counter values and sealing objects to disk.
Our first proposal is to ensure the atomicity of the \textit{write\_counter()} function
and \textit{seal\_object()} function. Currently in ROTE, the crash may happen between the 
two functions, contributing to a counter with a future value and making the KVS unrecoverable.
If we ensure the atomicity of the two functions, then the counter sealing and object sealing 
will succeed or fail at the same time, and the scenario where the counter has a future value 
will not appear.

Our second approach is to backup the counter value in disk, right before the verification 
of received counters in ROTE. In that case, if the crash happens after the sealing of counter
(before sealing the object), the KVS can still recover to the older version of counter value 
by referring to the log.

\subsection{Speicher}

The Speicher system~\cite{} is another KVS that provides rollback protection.
It mainly has two differences compared to ROTE introduced above. First, 
Speicher is a typical \textit{store-then-inc} based system that first stores 
object with pre-incremented counter values into persistent disk memory and increments
counter value thereafter. The second difference lies in the architecture. The ROTE system
uses several assisting severs organized by a trusted group manager while the Speicher system 
uses only localized protection.


\subsubsection{Overview of Speicher}


The update stage of Speicher works as follows. To update a record or add a new 
record in KVS, Speicher first inserts the record into a runtime data structure
named as \textit{memtable}. Then the counter is incremented to the value in 
the stored record. Speicher is a fully asynchronous system that decouples the 
increment operation and the store operation, which means that Speicher batches
the increment operation after a \textit{expected time}. During the expected 
time, the user can wait for the increment of counter to be stable, and can 
abort the update operation if time expires. This mechanism prevents users 
from potential rollback attacks. In case a system crashes between a store
operation and increment operation, the stored record will become invalid as 
the counter value is beyond the older version of the counter stored in disk.




\subsubsection{System Description}






\subsubsection{Limitations and Future Directions}

We claim that the design of Speicher does improve rollback protection and is robust to 
restore at any time of protocol execution. As an SGX based KVS, Speicher also considers 
the limited size of enclaves (around 128MB), and designed a new data structure outside 
the enclave to preserve the integrity. Unfortunately, Speicher may have failed to breakdown 
the overhead by each SGX components. For example, the asynchronous trusted counter mechanism
batches arrived counter increment operations and we expect to see the breakdown to see how 
this mechanism helps improve the throughput.



% \section{Runtime}
This is runtime.
\section{Evaluation Plan}
This is evaluation part.
\section{Conclusion}
In this report, we discussed the freshness problem in state-of-art Key-Value Store (KVS) 
systems. Typically, so-called rollback attackers might return an older version of records 
to users. The use of Intel SGX helps nullify part of this problem, including the 
confidentiality of data and integrity of execution code. However, directly apply SGX
cannot solve this problem because SGX also fails to check the latest version of records. 
Some prior works develop different counter increment based techniques to protect against
rollback attacks, including ROTE~\cite{}, Speicher~\cite{}, EnclaveDB~\cite{} and so on.
These systems either incurring moderate overhead or being effective in a distributed setting,
with some drawbacks in either evaluation or crash tolerance. 
Thus, we propose some improvements on the design and evaluation of existing work and plan 
to implement and evaluate our prototype.

%ACKNOWLEDGMENTS are optional
% \section{Acknowledgments}
% This section is optional.
%
% The following two commands are all you need in the
% initial runs of your .tex file to
% produce the bibliography for the citations in your paper.

\newpage
\bibliographystyle{abbrv}
\bibliography{sigproc-sp-csis8101}  % sigproc-sp-csis8101.bib is the name of the Bibliography in this case

% That's all folks!
\end{document}
