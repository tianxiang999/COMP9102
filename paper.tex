% This is sigproc-sp.tex -FILE FOR V2.6SP OF ACM_PROC_ARTICLE-SP.CLS
% OCTOBER 2002
%
% It is an example file showing how to use the 'acm_proc_article-sp.cls' V2.6SP
% LaTeX2e document class file for Conference Proceedings submissions.
% ----------------------------------------------------------------------------------------------------------------
% This .tex file (and associated .cls V2.6SP) *DOES NOT* produce:
%       1) The Permission Statement
%       2) The Conference (location) Info information
%       3) The Copyright Line with ACM data
%       4) Page numbering
%
%  However, both the CopyrightYear (default to 2002) and the ACM Copyright Data
% (default to X-XXXXX-XX-X/XX/XX) can still be over-ridden by whatever the author
% inserts into the source .tex file.
% e.g.
% \CopyrightYear{2003} will cause 2003 to appear in the copyright line.
% \crdata{0-12345-67-8/90/12} will cause 0-12345-67-8/90/12 to appear in the copyright line.
%
% ---------------------------------------------------------------------------------------------------------------
% It is an example which *does* use the .bib file (from which the .bbl file
% is produced).
% REMEMBER HOWEVER: After having produced the .bbl file,
% and prior to final submission,
% you need to 'insert'  your .bbl file into your source .tex file so as to provide
% ONE 'self-contained' source file.
%
% Questions regarding SIGS should be sent to
% Adrienne Griscti ---> griscti@acm.org
%
% Questions/suggestions regarding the guidelines, .tex and .cls files, etc. to
% Gerald Murray ---> murray@acm.org
%
% For tracking purposes - this is V2.6SP - OCTOBER 2002

\documentclass{acm_proc_article-csis8101}

\begin{document}
%
% --- Author Metadata here ---
\conferenceinfo{CSIS 8101 Surveys}{2012, CS Dept, HKU}
%\setpagenumber{50}
%\CopyrightYear{2002} % Allows default copyright year (2002) to be over-ridden - IF NEED BE.
%\crdata{0-12345-67-8/90/01}  % Allows default copyright data (X-XXXXX-XX-X/XX/XX) to be over-ridden.
% --- End of Author Metadata ---

\title{A Sample {\ttlit ACM} SIG Proceedings Paper in LaTeX
Format\titlenote{(Produces the permission block, copyright information and page numbering). For use with ACM\_PROC\_ARTICLE-SP.CLS V2.6SP. Supported by ACM.}}
%
% You need the command \numberofauthors to handle the "boxing"
% and alignment of the authors under the title, and to add
% a section for authors number 4 through n.

\author{
Ke Zhang\\
3030XXX\\
Department of Computer Science\\
Pokfulam Road, Hong Kong\\
\texttt{author1@cs.hku.hk}
\and Tianxiang Shen\\
3030058776\\
University of Hong Kong \\
Pokfulam Road, Hong Kong \\
\texttt{txshen2@cs.hku.hk}
}

\numberofauthors{2}
%
\maketitle
\begin{abstract}
This paper provides a sample of a \LaTeX\ document which conforms to
the formatting guidelines for ACM SIG Proceedings.
It complements the document \textit{Author's Guide to Preparing
ACM SIG Proceedings Using \LaTeX$2_\epsilon$\ and Bib\TeX}. This
source file has been written with the intention of being
compiled under \LaTeX$2_\epsilon$\ and BibTeX.

The developers have tried to include every imaginable sort
of ``bells and whistles", such as a subtitle, footnotes on
title, subtitle and authors, as well as in the text, and
every optional component (e.g. Acknowledgments, Additional
Authors, Appendices), not to mention examples of
equations, theorems, tables and figures.

To make best use of this sample document, run it through \LaTeX\
and BibTeX, and compare this source code with the printed
output produced by the dvi file.
\end{abstract}



\section{Introduction}
With the growth in cloud computing adoption, online data stored in data centers is growing at an ever 
increasing rate~\cite{bhatotia2012shredder}. Modern online services ubiquitously use persistent key-value (KV) store systems to store data with a high degree of reliability and performance~\cite{bailleu2019speicher}. Therefore, persistent KV stores have become a fundamental part of the cloud infrastructure.

At the same time, the risks of security violations in storage systems have increased significantly for the third-party cloud computing infrastructure~\cite{santos2009towards}. Modern data processing services hosted in cloud environments are under constant attack from malicious entities such as database administrators, server administrators, hackers who exploit bugs in the operating system or hypervisor, and even nation states. This results in frequent data breaches that reduce trust in online services. Semantically secure encryption can provide strong and efficient protection for data at rest and in transit, but this is not sufficient because data processing systems decrypt sensitive data in memory during query processing. In an untrusted environment, an attacker can compromise the security properties of the stored data and query operations. In fact, many studies show that software bugs, configuration errors, and security vulnerabilities pose a serious threat to storage systems~\cite{gunawi2014bugs}.

It is quite challenging to secure a storage system because modern storage systems are quite complex~\cite{lu2013study}. Thereby, the enforcement of security policies needs to be carried out by various layers in the system stack, which could expose the data to security vulnerabilities. Furthermore, since the data is stored outside the control of the data owner, the third-party storage platform provides an additional attack vector. The clients currently have limited support to verify whether the third-party operator, even with good intentions, can handle the data with the stated security guarantees.

An approach to enable secure query processing is trusted execution environments (TEEs), such as Intel Software Guard Extensions 
(SGX)~\cite{sgxexplained} or ARM TrustZone~\cite{winter2008trusted}, which provide an appealing approach to build secure systems. Enclaves can protect sensitive data and code, even from powerful attackers that control or have compromised the operating system and the hypervisor on a host machine. While enclaves can mitigate several attacks, using them requires careful refactoring of applications into trusted and untrusted components to achieve desired security and privacy goals. Furthermore, ensuring high level security properties such as confidentiality, integrity, and freshness requires additional logic to protect secrets when they leave the enclave and verify their integrity when they are read. This task is relatively simple in applications such as password checkers, key management systems and simpler data processing frameworks. In fact, given the importance of security threats in the cloud, there is a recent surge in leveraging TEEs for shielded execution of applications in the untrusted infrastructure. Shielded execution aims to provide strong security properties using a hardware-protected secure memory region or enclave.

While SGX can be considered as a big step forward towards trustworthy cloud computing, some attack vectors nevertheless remain. One important open issue are rollback and forking attacks on stateful applications that make use of persistent storage. Whereas SGX provides mechanisms against main- memory replay attacks, persistent storage is not under the direct control of SGX and therefore harder to secure. The need to handle system restarts, operating system crashes, and power outages makes a completely secure solution for state continuity difficult to achieve. 

In this report, we mainly survey three state-of-art methods in solving the secure key-value storage with rollback attacks protections, \textit{i.e.}, Monotonic Counter for SGX, ROTE system, and Speicher system. We separately describe their inner designations and their limitations together with our insights for future directions.

\section{Background}
\subsection{Secure Enclave}
A secure enclave is a set of software and hardware features that together provide an isolated execution environment to enable a set of strong security guarantees for applications running inside the enclave. Enclave allows user-level as well as Operating System~(OS) code to define private regions of memory, whose contents are protected and unable to be either read or saved by any process outside the enclave itself, including processes running at higher privilege levels~\cite{}. 

Primally, secure enclaves can provide confidentiality, integrity, and attestation. Confidentiality guarantees that an adversary outside of the enclave cannot inspect the state of execution inside the enclave, even if they compromise the operating system or correctness of the computation running inside the enclave even if the operating system has been compromised or a user attempts to subvert the execution of the program inside the enclave. Finally, hardware-based attestation provides an unforgeable proof that enables a remote party to verify what has run inside the enclave even if they don’t have physical access to the machine. A secure enclave thus provides a powerful cornerstone for secure computing and development of secure systems in general. 


Intuitively, secure enclave fundamentally ensures the correctness and isolation in executing given process. The confirmation of input data freshness is hard to achieve, especially when the enclave encounters crash or restart. There are several wildly used secure enclave services~\cite{}, one of the most popular security architectures is Intel Software Guard Extensions (SGX)~\cite{}. However, a mature secure enclave designation as SGX still shows unsatisfied performance towards rollback attacks. In this report, we focus on the SGX architecture and its existing promotions in proposing protection against rollback attacks.


\subsection{Rollback Attack}

Rollback attacks remain a potential secure problem in secure enclave. In a rollback attack, attackers replace the latest data with an older version without being identified by the system. 

Data integrity violation through rollback attacks can have severe implications. Consider, for example, a financial application implemented as an enclave. The enclave repeatedly processes incoming transactions at high speed and maintains an account balance for each user or a history of all transactions in the system. If the adversary manages to revert the enclave to its previous state, the maintained account balance or the queried transaction history does not match the executed transactions.

In reality, enclaves cannot easily detect this replay, because the processor is unable to maintain persistent state across enclave executions that may include reboots or crash. Another way to carry out rollback attacks in secure enclaves is to create multiple instances of a same process and route update requests to one instance and read requests to the other. Due to the characteristic of secure enclave, the instances are indistinguishable to remote clients or OS.

To avoid rollback attacks, most commonly considered direction is to record the time related information for every state change. In this paper, we mainly discuss three designations in rollback attacks protection built on the SGX architecture.


\section{Overview}
This is overview.
\section{Runtime}
This is runtime.
\section{Evaluation Plan}
This is evaluation part.
\section{Conclusion}
In this report, we discussed the freshness problem in state-of-art Key-Value Store (KVS) 
systems. Typically, so-called rollback attackers might return an older version of records 
to users. The use of Intel SGX helps nullify part of this problem, including the 
confidentiality of data and integrity of execution code. However, directly apply SGX
cannot solve this problem because SGX also fails to check the latest version of records. 
Some prior works develop different counter increment based techniques to protect against
rollback attacks, including ROTE~\cite{matetic2017rote}, Speicher~\cite{bailleu2019speicher}, EnclaveDB~\cite{enclavedb:sp18} and so on.
These systems either incurring moderate overhead or being effective in a distributed setting,
with some drawbacks in either evaluation or crash tolerance. 
Thus, we propose some improvements on the design and evaluation of existing work and plan 
to implement and evaluate our prototype.

%ACKNOWLEDGMENTS are optional
\section{Acknowledgments}
This section is optional.
%
% The following two commands are all you need in the
% initial runs of your .tex file to
% produce the bibliography for the citations in your paper.
\bibliographystyle{abbrv}
\bibliography{sigproc-sp-csis8101}  % sigproc-sp-csis8101.bib is the name of the Bibliography in this case

% That's all folks!
\end{document}
